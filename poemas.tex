\documentclass[10pt,a5paper,oneside]{book}

\usepackage[brazilian]{babel}
\usepackage[utf8]{inputenc}
\usepackage[T1]{fontenc}
\usepackage{lmodern, parskip, titletoc, emptypage}
\usepackage[hyphens]{url}
\usepackage[hidelinks]{hyperref}
\usepackage[pagestyles]{titlesec}

\usepackage[inner=1in, outer=0.5in, top=0.5in, bottom=0.7in, head=0.25in, foot=0.35in]{geometry}

% paragraphs
\setlength{\parindent}{0em}
\setlength{\parskip}{1em}

% titles
\titleformat{\part}[display]{\centering\bfseries\Large}{\thepart}{1em}{}
\titleformat{\chapter}{\bfseries\large}{\thechapter}{1em}{}

% toc
\titlecontents{part}[0em]{\addvspace{0.5em}}{}{\bfseries}{}{\addvspace{1em}}
\titlecontents{chapter}[1em]{}{}{}{\hfill\contentspage}{}

% pagestyle
\renewpagestyle{plain}{
    \setfoot[\footnotesize\thepage][][]{}{}{\footnotesize\thepage}
}
\pagestyle{plain}

%===============================================================================
\begin{document}

%-------------------------------------------------------------------------------
\pagestyle{empty}

\title{Poemas}
\author{}
\date{}
\maketitle

\frontmatter
\tableofcontents

\mainmatter

%===============================================================================
\part{Manuel Maria Barbosa du Bocage}

%-------------------------------------------------------------------------------
\chapter{Contra os que negam o livre-arbítrio nas acções humanas}

\begin{verse}
Vós, crédulos mortais, alucinados\\
de sonhos, de quimeras, de aparências,\\
colheis por uso erradas consequências\\
dos acontecimentos desastrados:

Se à perdição correis precipitados\\
por cegas, por fogosas impaciências,\\
indo a cair, gritais que são violências\\
d'inexoráveis Céus, de negros fados:

Se um celeste poder tirano, e duro,\\
às vezes extorquisse as liberdades,\\
Que prestava, ó Razão, teu lume puro?

Não forçam corações as divindades;\\
fado amigo não há, nem fado escuro:\\
fados são as paixões, são as vontades.
\end{verse}

%-------------------------------------------------------------------------------
\chapter{Meu ser evaporei na lida insana}

\begin{verse}
Meu ser evaporei na lida insana\\
do tropel de paixões que me arrastava;\\
ah! cego eu cria, ah! mísero eu sonhava\\
em mim quase imortal a essência humana.

De que inúmeros sóis a mente ufana\\
existência falaz me não dourava!\\
Mas eis sucumbe a Natureza escrava\\
ao mal, que a vida em sua orgia dana.

Prazeres, sócios meus, e meus tiranos!\\
Esta alma, que sedenta em si não coube,\\
no abismo vos sumiu dos desenganos.

Deus, oh Deus!\ldots{} Quando a morte à luz me roube,\\
ganhe um momento o que perderam anos,\\
saiba morrer o que viver não soube.
\end{verse}

%-------------------------------------------------------------------------------
\chapter{Já Bocage não sou!\ldots{} À cova escura}

\begin{verse}
Já Bocage não sou!\ldots{} À cova escura\\
meu estro vai parar desfeito em vento\ldots{}\\
Eu aos Céus ultrajei! O meu tormento\\
leve me torne sempre a terra dura.

Conheço agora já quão vã figura\\
em prosa e verso fez meu louco intento.\\
Musa!\ldots{} Tivera algum merecimento,\\
se um raio da razão seguisse, pura!

Eu me arrependo; a língua quase fria\\
brade em alto pregão à mocidade,\\
que atrás do som fantástico corria:

Outro Aretino fui\ldots{} A santidade\\
manchei!\ldots{} Oh!, se me creste, gente ímpia,\\
rasga meus versos, crê na Eternidade!
\end{verse}

%===============================================================================
\part{Luís Vaz de Camões}

%-------------------------------------------------------------------------------
\chapter{Soneto 000 --- Eu cantei já, e agora vou chorando}

\begin{verse}
Eu cantei já, e agora vou chorando\\
o tempo que cantei tão confiado;\\
parece que no canto já passado\\
se estavam minhas lágrimas criando.

Cantei: mas se me alguém pergunta ``quando'':\\
não sei, que também fui nisso enganado.\\
É tão triste este meu presente estado\\
que o passado por ledo estou julgando.

Fizeram-me cantar, manhosamente,\\
contentamentos não, mas confianças;\\
cantava, mas já era ao som dos ferros.

De quem me queixarei, se tudo mente?\\
mas eu que culpa ponho às esperanças\\
onde a Fortuna injusta é mais que os erros?
\end{verse}

%-------------------------------------------------------------------------------
\chapter{Soneto 004 --- Tanto de meu estado me acho incerto}

\begin{verse}
Tanto de meu estado me acho incerto,\\
que em vivo ardor tremendo estou de frio;\\
sem causa, justamente choro e rio,\\
o mundo todo abarco e nada aperto.

É tudo quanto sinto, um desconcerto;\\
da alma um fogo me sai, da vista um rio;\\
agora espero, agora desconfio,\\
agora desvario, agora acerto.

Estando em terra, chego ao Céu voando;\\
numa hora acho mil anos, e é de jeito\\
que em mil anos não posso achar uma hora.

Se me pergunta alguém por que assim ando,\\
respondo que não sei; porém suspeito\\
que só porque vos vi, minha Senhora.
\end{verse}

%-------------------------------------------------------------------------------
\chapter{Soneto 007 --- O fogo que na branda cera ardia}

\begin{verse}
O fogo que na branda cera ardia,\\
vendo o rosto gentil que eu nalma vejo,\\
se acendeo noutro fogo de desejo,\\
por alcançar a luz que vence o dia.

Como em dous ardores se acendia,\\
da grande impaciência fez desejo,\\
e, remetendo com furor sobejo,\\
vos foi beijar na face em que se via.

Ditosa aquela flama que se atreve\\
a pagar seus ardores e tormentos,\\
na vista de que o mundo tremer deve.

Namoram-se, Senhora, os Elementos\\
de vós; e queima o fogo aquela neve,\\
que abrasa corações e pensamentos.
\end{verse}

%-------------------------------------------------------------------------
\chapter{Soneto 013 --- Alegres campos, verdes arvoredos}

\begin{verse}
Alegres campos, verdes arvoredos,\\
claras e frescas águas de cristal,\\
que em vós os debuxais ao natural,\\
discorrendo da altura dos rochedos;

Silvestres montes, ásperos penedos,\\
compostos em concerto desigual:\\
sabei que, sem licença de meu mal,\\
já não podeis fazer meus olhos ledos.

E pois me já não vedes como vistes,\\
não me alegrem verduras deleitosas\\
nem águas que correndo alegres vêm.

Semearei em vós lembranças tristes,\\
regando-vos com lágrimas saudosas,\\
e nascerão saudades de meu bem.
\end{verse}

%-------------------------------------------------------------------------------
\chapter{Soneto 020 --- Transforma-se o amador na cousa amada}

\begin{verse}
Transforma-se o amador na cousa amada,\\
por virtude do muito imaginar;\\
não tenho, logo, mais que desejar,\\
pois em mim tenho a parte desejada.

Se nela está minha alma transformada,\\
que mais deseja o corpo de alcançar?\\
Em si somente pode descansar,\\
pois consigo tal alma está liada.

Mas esta linda e pura semideia,\\
que, como um acidente em seu sujeito,\\
assi co a alma minha se conforma,

está no pensamento como ideia:\\
e o vivo e puro amor de que sou feito,\\
como a matéria simples busca a forma.
\end{verse}

%-------------------------------------------------------------------------
\chapter{Soneto 025 --- Oh! Como se me alonga de ano em ano}

\begin{verse}
Oh! Como se me alonga, de ano em ano,\\
a peregrinação cansada minha!\\
Como se encurta e como ao fim caminha\\
este meu breve e vão discurso humano!

Vai-se gastando a idade e cresce o dano;\\
perde-se-me um remédio que 'inda tinha;\\
se por experiência se adivinha,\\
qualquer grande esperança é grande engano.

Corro após este bem que não se alcança;\\
no meio do caminho me falece;\\
mil vezes caio, e perco a confiança.

Quando ele foge, eu tardo; e, na tardança,\\
se os olhos ergo, a ver se inda parece,\\
da vista se me perde, e da esperança.
\end{verse}

%-------------------------------------------------------------------------
\chapter{Soneto 047 --- Oh! quão caro me custa o entender-te}

\begin{verse}
Oh! quão caro me custa o entender-te,\\
molesto Amor, que, só por alcançar-te,\\
de dor em dor me tens trazido a parte\\
onde em ti ódio e ira se converte!

Cuidei que, para em tudo conhecer-te,\\
me não faltasse experiência e arte;\\
agora vejo na alma acrescentar-te\\
aquilo que era causa de perder-te.

Estavas tão secreto no meu peito,\\
que eu mesmo, que te tinha, não sabia\\
que me senhoreavas deste jeito.

Descobriste-te agora; e foi por via\\
que teu descobrimento e meu defeito,\\
um me envergonha e outro me injuria.
\end{verse}

%-------------------------------------------------------------------------------
\chapter{Soneto 080 --- Alma minha gentil, que te partiste}

\begin{verse}
Alma minha gentil, que te partiste\\
tão cedo desta vida descontente,\\
repousa lá no Céu eternamente\\
e viva eu cá na terra sempre triste.

Se lá no assento etéreo, onde subiste,\\
memória desta vida se consente,\\
não te esqueças daquele amor ardente\\
que já nos olhos meus tão puro viste.

E se vires que pode merecer-te\\
alguma cousa a dor que me ficou\\
da mágoa, sem remédio, de perder-te,

roga a Deus, que teus anos encurtou,\\
que tão cedo de cá me leve a ver-te,\\
quão cedo de meus olhos te levou.
\end{verse}

%-------------------------------------------------------------------------
\chapter{Soneto 095 --- Ondados fios de ouro reluzente}

\begin{verse}
Ondados fios de ouro reluzente,\\
que, agora da mão bela recolhidos,\\
agora sobre as rosas estendidos,\\
fazeis que sua beleza se acrescente;

Olhos, que vos moveis tão docemente,\\
em mil divinos raios incendidos,\\
se de cá me levais alma e sentidos,\\
que fôra, se de vós não fôra ausente?

Honesto riso, que entre a mor fineza\\
de perlas e corais nasce e parece,\\
se na alma em doces ecos não o ouvisse;

Se, imaginando só tanta beleza,\\
de si em nova glória a alma se esquece,\\
que será quando a vir? Ah! quem a visse!
\end{verse}

%-------------------------------------------------------------------------
\chapter{Soneto 104 --- Correm turvas as águas deste rio}

\begin{verse}
Correm turvas as águas deste rio,\\
que as do Céu e as do monte as enturbaram;\\
os campos florecidos se secaram,\\
intratável se fez o vale, e frio.

Passou o verão, passou o ardente estio,\\
umas coisas por outras se trocaram;\\
os fementidos Fados já deixaram\\
do mundo o regimento, ou desvario.

Tem o tempo sua ordem já sabida;\\
o mundo, não; mas anda tão confuso,\\
que parece que dele Deus se esquece.

Casos, opiniões, natura e uso\\
fazem que nos pareça desta vida\\
que não há nela mais que o que parece.
\end{verse}

%-------------------------------------------------------------------------
\chapter{Soneto 108 --- Erros meus, má fortuna, amor ardente}

\begin{verse}
Erros meus, má fortuna, amor ardente\\
em minha perdição se conjuraram;\\
os erros e a fortuna sobejaram,\\
que para mim bastava o amor somente.

Tudo passei; mas tenho tão presente\\
a grande dor das cousas que passaram,\\
que as magoadas iras me ensinaram\\
a não querer já nunca ser contente.

Errei todo o discurso de meus anos;\\
dei causa que a Fortuna castigasse\\
as minhas mal fundadas esperanças.

De amor não vi senão breves enganos.\\
Oh! quem tanto pudesse que fartasse\\
este meu duro génio de vinganças!
\end{verse}

%-------------------------------------------------------------------------------
\chapter{Soneto 144 --- Sustenta meu viver uma esperança}

\begin{verse}
Sustenta meu viver uma esperança\\
derivada de um bem tão desejado\\
que, quando nela estou mais confiado,\\
mor dúvida me põe qualquer mudança.

E quando inda este bem na mor pujança\\
de seus gostos me tem mais enlevado,\\
me atormenta então ver eu que, alcançado\\
será por quem de vós não tem lembrança.

Assi que, nestas redes enlaçado,\\
a penas dou a vida, sustentando\\
uma nova matéria a meu cuidado,

suspiros d'alma tristes arrancando,\\
dos silvos de uma pedra acompanhado,\\
estou matérias tristes lamentando.
\end{verse}

%-------------------------------------------------------------------------------
\chapter{Soneto 165 --- Vós outros, que buscais repouso certo}

\begin{verse}
Vós outros, que buscais repouso certo\\
na vida, com diversos exercícios;\\
a quem, vendo do mundo os benefícios,\\
o regimento seu está encoberto;

Dedicai, se quereis, ao desconcerto\\
novas honras e cegos sacrifícios;\\
que, por castigo igual de antigos vícios,\\
quer Deus que andem as cousas por acerto.

Não caiu neste modo de castigo\\
quem pôs culpa à Fortuna, quem somente\\
crê que acontecimentos há no mundo.

A grande experiência é grão perigo;\\
mas o que a Deus é justo e evidente\\
parece injusto aos homens e profundo.
\end{verse}

%-------------------------------------------------------------------------------
\chapter{Ao Desconcerto do Mundo}

\begin{verse}
Os bons vi sempre passar\\
no Mundo grandes tormentos;\\
e para mais me espantar,\\
os maus vi sempre nadar\\
em mar de contentamentos.\\
Cuidando alcançar assim\\
o bem tão mal ordenado,\\
fui mau, mas fui castigado:\\
assim que, só para mim,\\
anda o Mundo concertado.
\end{verse}

%===============================================================================
\part{Juvenal Galeno}

%-------------------------------------------------------------------------------
\chapter{A Jangada}

\begin{verse}
Minha jangada de vela,\\
Que vento queres levar?\\
Tu queres vento de terra,\\
Ou queres vento do mar?\\
Minha jangada de vela,\\
Que vento queres levar?

Aqui no meio das ondas,\\
Das verdes ondas do mar,\\
És como que pensativa,\\
Duvidosa a bordejar!\\
Minha jangada de vela,\\
Que vento queres levar?

Saudades tens lá das praias,\\
Queres n'areia encalhar?\\
Ou no meio do oceano\\
Apraz-te as ondas sulcar?\\
Minha jangada de vela,\\
Que vento queres levar?

Sobre as vagas, como a garça,\\
Gosto de ver-te adejar,\\
Ou qual donzela no prado\\
Resvalando a meditar:\\
Minha jangada de vela,\\
Que vento queres levar?

Se a fresca brisa da tarde\\
A vela vem te oscular,\\
Estremeces como a noiva\\
Se vem-lhe o noivo beijar:\\
Minha jangada de vela,\\
Que vento queres levar?

Quer sossegada na praia,\\
Quer nos abismos do mar,\\
Tu és, ó minha jangada,\\
A virgem do meu sonhar:\\
Minha jangada de vela,\\
Que vento queres levar?

Sé à liberdade suspiro,\\
Vens liberdade me dar;\\
Se fome tenho --- ligeira\\
Me trazes para pescar!\\
Minha jangada de vela,\\
Que vento queres levar?

A tua vela branquinha\\
Acabo de borrifar;\\
Já peixe tenho de sobra,\\
Vamos à terra aproar:\\
Minha jangada de vela,\\
Que vento queres levar?

Ai, vamos, que as verdes ondas,\\
Fagueiras a te embalar,\\
São falsas nestas alturas\\
Quais lá na beira do mar:\\
Minha jangada de vela,\\
Que vento queres levar?
\end{verse}

%===============================================================================
\part{Casimiro de Abreu}

%-------------------------------------------------------------------------------
\chapter{Meus oito anos}

\begin{verse}
Oh! que saudades que eu tenho\\
Da aurora da minha vida,\\
Da minha infância querida\\
Que os anos não trazem mais!\\
Que amor, que sonhos, que flores,\\
Naquelas tardes fagueiras\\
À sombra das bananeiras,\\
Debaixo dos laranjais!

Como são belos os dias\\
Do despontar da existência!\\
-- Respira a alma inocência\\
Como perfumes a flor;\\
O mar é --- lago sereno,\\
O céu --- um manto azulado,\\
O mundo --- um sonho dourado,\\
A vida --- um hino d'amor!

Que auroras, que sol, que vida,\\
Que noites de melodia\\
Naquela doce alegria,\\
Naquele ingênuo folgar!\\
O céu bordado d'estrelas,\\
A terra de aromas cheia,\\
As ondas beijando a areia\\
E a lua beijando o mar!

Oh! dias de minha infância!\\
Oh! meu céu de primavera!\\
Que doce a vida não era\\
Nessa risonha manhã!\\
Em vez de mágoas de agora,\\
Eu tinha nessas delícias\\
De minha mãe as carícias\\
E beijos de minha irmã!

Livre filho das montanhas,\\
Eu ia bem satisfeito,\\
De camisa aberta ao peito,\\
-- Pés descalços, braços nus --\\
Correndo pelas campinas\\
À roda das cachoeiras,\\
Atrás das asas ligeiras\\
Das borboletas azuis!

Naqueles tempos ditosos\\
Ia colher as pitangas,\\
Trepava a tirar as mangas,\\
Brincava à beira do mar;\\
Rezava às Ave-Marias,\\
Achava o céu sempre lindo,\\
Adormecia sorrindo,\\
E despertava a cantar!

Oh! que saudades que eu tenho\\
Da aurora da minha vida\\
Da minha infância querida\\
Que os anos não trazem mais!\\
-- Que amor, que sonhos, que flores,\\
Naquelas tardes fagueiras\\
À sombra das bananeiras,\\
Debaixo dos laranjais!
\end{verse}

%===============================================================================
\part{Alberto Caeiro}

%-------------------------------------------------------------------------------
\chapter{XXIV --- O que Nós Vemos}

\begin{verse}
O que nós vemos das cousas são as cousas.\\
Por que veríamos nós uma cousa se houvesse outra?\\
Por que é que ver e ouvir seria iludirmo-nos\\
Se ver e ouvir são ver e ouvir?

O essencial é saber ver,\\
Saber ver sem estar a pensar,\\
Saber ver quando se vê,\\
E nem pensar quando se vê\\
Nem ver quando se pensa.

Mas isso (tristes de nós que trazemos a alma vestida!),\\
Isso exige um estudo profundo,\\
Uma aprendizagem de desaprender\\
E uma sequestração na liberdade daquele convento\\
De que os poetas dizem que as estrelas são as freiras eternas\\
E as flores as penitentes convictas de um só dia,\\
Mas onde afinal as estrelas não são senão estrelas\\
Nem as flores senão flores.\\
Sendo por isso que lhes chamamos estrelas e flores.
\end{verse}

%===============================================================================
\part{Ângelo Monteiro}

%-------------------------------------------------------------------------------
\chapter{Em nosso sol se instala o Inimigo}

\begin{verse}
Em nosso sol se instala o Inimigo\\
se afirma em cada passo que o renega\\
em nossas águas seu negror navega\\
se esconde em nossa vinha e em nosso trigo.

Ao próprio Amor domina o Inimigo\\
ao torná-lo mais dócil em paixão cega.\\
A nossa inteira vida a ele se apega\\
e contra ele é vão qualquer abrigo.

Das nossas próprias ânsias se sustenta\\
dos nossos próprios gritos se arrebata\\
e sua teia é tão longa quanto lenta.

Sempre belo se veste de gerânios\\
e jovem como a luz --- que não tem data\\
--- nos sepulta nos seus subterrâneos.
\end{verse}

%===============================================================================
\part{Fernando Pessoa}

%-------------------------------------------------------------------------------
\chapter{Presságio}

\begin{verse}
O amor, quando se revela,\\
Não se sabe revelar.\\
Sabe bem olhar pra ela,\\
Mas não lhe sabe falar.

Quem quer dizer o que sente\\
Não sabe o que há de dizer.\\
Fala: parece que mente\ldots{}\\
Cala: parece esquecer\ldots{}

Ah, mas se ela adivinhasse,\\
Se pudesse ouvir o olhar,\\
E se um olhar lhe bastasse\\
Pra saber que a estão a amar!

Mas quem sente muito, cala;\\
Quem quer dizer quanto sente\\
Fica sem alma nem fala,\\
Fica só, inteiramente!

Mas se isto puder contar-lhe\\
O que não lhe ouso contar,\\
Já não terei que falar-lhe\\
Porque lhe estou a falar\ldots{}
\end{verse}

%-------------------------------------------------------------------------------
\chapter{Os deuses vendem quando dão}

\begin{verse}
Os Deuses vendem quando dão.\\
Compra-se a glória com desgraça.\\
Ai dos felizes, porque são\\
Só o que passa!

Baste a quem baste o que lhe basta\\
O bastante de lhe bastar!\\
A vida é breve, a alma é vasta:\\
Ter é tardar.

Foi com desgraça e com vileza\\
Que Deus ao Cristo definiu:\\
Assim o opôs à Natureza\\
E Filho o ungiu.
\end{verse}

%===============================================================================
\part{Alphonsus de Guimarães Filho}

%-------------------------------------------------------------------------------
\chapter{Coágulo}

\begin{verse}
De repente direi tudo.\\
Mas com tanta veemência\\
e com tamanha aspereza\\
de expressão e sofrimento,\\
que terás minha demência\\
no coágulo sangrento\\
desabado sobre a mesa.

E sairei pelas ruas\\
sem saber em que cidade\\
estive, estou, estarei.\\
Triste alegre puro impuro\\
vejo a morte em cada muro\\
a morte na campainha\\
ressoando do outro lado.\\
E estertorando direi\\
que vejo sangue pisado\\
nessas ervas pés e mãos\\
nesses gestos nesses risos\\
que vejo sangue pisado\\
até na face do Rei!

De repente, num soluço,\\
direi tudo quanto existe;\\
não serei nem bom nem triste.\\
Serei apenas um grito\\
doloroso rebentado\\
na convulsão de um momento.\\
E o mundo penoso aflito\\
restará desesperado\\
num coágulo sangrento.
\end{verse}

%-------------------------------------------------------------------------------
\chapter{Náufrago}

\begin{verse}
E temo, e temo tudo, e nem sei o que temo.\\
Perde-se o meu olhar pelas trevas sem fim.\\
Medonha é a escuridão do céu, de extremo a extremo\ldots{}\\
De que noite sem luar, mísero e triste, vim?

Amedronta-me a terra, e se a contemplo, tremo.\\
Que mistério fatal corveja sobre mim?\\
E ao sentir-me no horror do caos, como um blasfemo,\\
Não sei por que padeço, e choro, e anseio assim.

A saudade tirita aos meus pés: vai deixando\\
Atrás de si a mágoa e o sonho\ldots{} E eu, miserando,\\
Caminho para a morte alucinado e só.

O naufrágio, meu Deus! Sou um navio sem mastros.\\
Como custa a minha alma a transformar-se em astros,\\
Como este corpo custa a desfazer-se em pó!
\end{verse}

%-------------------------------------------------------------------------------
\chapter{Do Azul, num soneto}

\begin{verse}
Verificar o azul nem sempre é puro.\\
Melhor será revê-lo entre as ramadas\\
e os altos frutos de um pomar escuro\\
--- azul de tênues bocas desoladas.

Melhor será sonhá-lo em madrugadas,\\
fresco, inconstante azul sempre imaturo,\\
azul de claridades sufocadas\\
latejando nas pedras --- nascituro.

Não este azul, mas outro e dolorido,\\
evanescente azul que na orvalhada\\
ficou, pétala ingênua, torturada.

Recupero-o, sem ter, e ei-lo perdido,\\
azul de voz, de sombra envenenada,\\
que em nós se esvai sem nunca ter vivido.
\end{verse}

%===============================================================================
\part{Augusto Meyer Jr.}

%-------------------------------------------------------------------------------
\chapter{Soneto I}

\begin{verse}
Gota de luz no cálice de agosto,\\
sabe a lúcida calma o desengano.\\
Em vão devora o tempo o mês e o ano:\\
vindima é a vida, vinho me é o Sol-posto.

Cobre-se o vale de um rubor humano.\\
Um beijo solto voa no ar, um gosto\\
de uva madura, um aroma de mosto\\
desce da rubra luz do céu serrano.

Vem, noite grave. E assim chegasse o outono\\
meu, tão sutil e manso como agora\\
mesmo subiu a sombra serra acima\ldots{}

Tudo se apague e a hora esqueça a hora,\\
que só do sonho eu vivo, e grato é o sono\\
a quem provou seu dia de vindima.
\end{verse}

%-------------------------------------------------------------------------------
\chapter{Soneto II}

\begin{verse}
A quem provou seu dia de vindima,\\
votado ao outro lado, ao eco, ao nada,\\
grata é a sombra mais longa e o fim da estrada,\\
começo de um descer, que é mais acima.

Grave, de uma tristeza inconsolada\\
mas fiel, minha sombra é minha rima.\\
Princípio de um além que se aproxima\\
é o fim, talvez limiar de outra morada.

Gosto amargo e tão doce de ter sido\\
poroso a tudo, alma aberta às auroras\\
que hão de nascer, e ao lembrado e esquecido!

Saudade! mas saudade em que não choras,\\
senão cantando, o próprio mal vivido\ldots{}\\
Que as horas voltem sempre, as mesmas horas!
\end{verse}

%===============================================================================
\part{Augusto dos Anjos}

%-------------------------------------------------------------------------------
\chapter{Coração Frio}

\begin{verse}
Frio o sagrado coração da lua,\\
Teu coração rolou da luz serena!\\
E eu tinha ido ver a aurora tua\\
Nos raios d'ouro da celeste arena\ldots{}

E vi-te triste, desvalida e nua!\\
E o olhar perdi, ansiando a luz amena\\
No silêncio notívago da rua\ldots{}\\
--- Sonâmbulo glacial da estranha pena!

Estavas fria! A neve que a alma corta\\
Não gele talvez mais, nem mais alquebre\\
Um coração como a alma que está morta\ldots{}

E estavas morta, eu vi, eu que te almejo,\\
Sombra de gelo que me apaga a febre,\\
--- Lua que esfria o sol do meu desejo!
\end{verse}

%===============================================================================
\part{Cruz e Souza}

%-------------------------------------------------------------------------------
\chapter{Sorriso Interior}

\begin{verse}
O ser que é ser e que jamais vacila\\
Nas guerras imortais entra sem susto,\\
Leva consigo este brasão augusto\\
Do grande amor, da grande fé tranquila.

Os abismos carnais da triste argila,\\
Ele os vence sem ânsias e sem custo\ldots{}\\
Fica sereno, num sorriso justo,\\
Enquanto tudo em derredor oscila.

Ondas interiores de grandeza\\
Dão-lhe esta glória em frente à Natureza,\\
Esse esplendor, todo esse largo eflúvio.

O ser que é ser transforma tudo em flores\ldots{}\\
E para ironizar as próprias dores\\
Canta por entre as águas do Dilúvio!
\end{verse}

%===============================================================================
\part{Manuel Bandeira}

%-------------------------------------------------------------------------------
\chapter{Soneto Plagiado de Augusto Frederico Schmidt}

\begin{verse}
E de súbito n'alma incompreendida\\
esta mágoa, esta pena, esta agonia;\\
nos olhos ressequidos a sombria\\
fonte de pranto, quente e irreprimida.

No espírito deserto, a impressentida\\
misteriosa presença que não via;\\
a consciência do mal que não sabia,\\
aparecida, desaparecida\ldots{}

Até bem pouco, era uma imagem baça.\\
Agora, neste instante de certeza,\\
surgindo claro, como nunca o vi!

E nesse olhar tocado pela graça\\
do céu, não sei que angélica pureza,\\
--- pureza que não tenho, que perdi.
\end{verse}

%-------------------------------------------------------------------------------
\chapter{Noturno da Mosela}

\begin{verse}
A noite\ldots{} O silêncio\ldots{}\\
Se fosse só o silêncio!\\
Mas esta queda d'água que não pára! que não pára!\\
Não é de dentro de mim que ela flui sem piedade?\ldots{}\\
A minha vida foge, foge --- e sinto que foge inutilmente!

O silêncio e a estrada ensopada, com dois reflexos intermináveis\ldots{}

Fumo até quase não sentir mais que a brasa e a cinza em minha boca.\\
O fumo faz mal aos meus pulmões comidos pelas algas.\\
O fumo é amargo e abjeto. Fumo abençoado, que és amargo e abjeto!

Uma pequenina aranha urde no peitoril da janela a teiazinha levíssima.

Tenho vontade de beijar esta aranhazinha\ldots{}

No entanto em cada charuto que acendo cuido encontrar o gosto que faz esquecer\ldots{}

Os meus retratos\ldots{} Os meus livros\ldots{} O meu crucifixo de marfim\ldots{}
E a noite\ldots{}
\end{verse}

%===============================================================================
\part{Antero de Quental}

%-------------------------------------------------------------------------------
\chapter{Transcendentalismo}

\begin{verse}
Já sossega, depois de tanta luta,\\
já me descansa em paz o coração.\\
Caí na conta, enfim, de quanto é vão\\
o bem que ao Mundo e à Sorte se disputa.

Penetrando, com fronte não enxuta,\\
no sacrário do templo da Ilusão,\\
só encontrei, com dor e confusão,\\
trevas e pó, uma matéria bruta\ldots{}

Não é no vasto Mundo --- por imenso\\
que ele pareça à nossa mocidade ---\\
que a alma sacia o seu desejo intenso\ldots{}

Na esfera do invisível, do intangível,\\
sobre desertos, vácuo, soledade,\\
voa e paira o espírito impassível!
\end{verse}

%-------------------------------------------------------------------------------
\chapter{Na mão de Deus}

\begin{verse}
Na mão de Deus, na Sua mão direita,\\
descansou afinal meu coração.\\
Do palácio encantado da Ilusão\\
desci a passo e passo a escada estreita.

Como as flores mortais, com que se enfeita\\
a ignorância infantil, despojo vão,\\
depus do Ideal e da Paixão\\
a forma transitória e imperfeita.

Como criança, em lôbrega jornada,\\
que a mãe leva no colo agasalhada\\
e atravessa, sorrindo vagamente,

selvas, mares, areias do deserto\ldots{}\\
dorme o teu sono, coração liberto,\\
dorme na mão de Deus eternamente!
\end{verse}

%-------------------------------------------------------------------------------
\chapter{Ignoto Deo}

\begin{verse}
Que beleza mortal se te assemelha,\\
Ó sonhada visão desta alma ardente,\\
que reflectes em mim teu brilho ingente,\\
lá como sobre o mar o Sol se espelha?

O Mundo é grande --- e esta ânsia me aconselha\\
a buscar-te na Terra: e eu, pobre crente,\\
pelo Mundo procuro um Deus clemente,\\
mas a ara só lhe encontro\ldots{} nua e velha\ldots{}

Não é mortal o que eu em ti adoro.\\
Que és tu aqui? olhar de piedade,\\
gota de mel em taça de venenos\ldots{}

Pura essência das lágrimas que choro\\
e sonho dos meus sonhos! se és verdade,\\
descobre-te, visão, no Céu ao menos!
\end{verse}

%===============================================================================
\part{Olavo Bilac}

%-------------------------------------------------------------------------------
\chapter{Este é o altivo pecador sereno}

\begin{verse}
Este é o altivo pecador sereno,\\
Que os soluços afoga na garganta,\\
E, calmamente, o copo de veneno\\
Aos lábios frios sem tremer levanta.

Tonto, no escuro pantanal terreno\\
Rolou. E, ao cabo de torpeza tanta,\\
Nem assim, miserável e pequeno,\\
Com tão grandes remorsos se quebranta.

Fecha a vergonha e as lágrimas consigo\ldots{}\\
E, o coração mordendo impenitente,\\
E, o coração rasgando castigado,

Aceita a enormidade do castigo,\\
Com a mesma face com que antigamente\\
Aceitava a delícia do pecado.
\end{verse}

%===============================================================================
\part{Álvaro de Campos}

%-------------------------------------------------------------------------------
\chapter{Tabacaria}

\begin{verse}
Não sou nada.\\
Nunca serei nada.\\
Não posso querer ser nada.\\
À parte isso, tenho em mim todos os sonhos do mundo.

Janelas do meu quarto,\\
Do meu quarto de um dos milhões do mundo que ninguém sabe quem é\\
(E se soubessem quem é, o que saberiam?),\\
Dais para o mistério de uma rua cruzada constantemente por gente,\\
Para uma rua inacessível a todos os pensamentos,\\
Real, impossivelmente real, certa, desconhecidamente certa,\\
Com o mistério das coisas por baixo das pedras e dos seres,\\
Com a morte a por umidade nas paredes e cabelos brancos nos homens,\\
Com o Destino a conduzir a carroça de tudo pela estrada de nada.

Estou hoje vencido, como se soubesse a verdade.\\
Estou hoje lúcido, como se estivesse para morrer,\\
E não tivesse mais irmandade com as coisas\\
Senão uma despedida, tornando-se esta casa e este lado da rua\\
A fileira de carruagens de um comboio, e uma partida apitada\\
De dentro da minha cabeça,\\
E uma sacudidela dos meus nervos e um ranger de ossos na ida.

Estou hoje perplexo, como quem pensou e achou e esqueceu.\\
Estou hoje dividido entre a lealdade que devo\\
À Tabacaria do outro lado da rua, como coisa real por fora,\\
E à sensação de que tudo é sonho, como coisa real por dentro.

Falhei em tudo.\\
Como não fiz propósito nenhum, talvez tudo fosse nada.\\
A aprendizagem que me deram,\\
Desci dela pela janela das traseiras da casa.\\
Fui até ao campo com grandes propósitos.\\
Mas lá encontrei só ervas e árvores,\\
E quando havia gente era igual à outra.\\
Saio da janela, sento-me numa cadeira. Em que hei de pensar?

Que sei eu do que serei, eu que não sei o que sou?\\
Ser o que penso? Mas penso tanta coisa!\\
E há tantos que pensam ser a mesma coisa que não pode haver tantos!\\
Gênio? Neste momento\\
Cem mil cérebros se concebem em sonho gênios como eu,\\
E a história não marcará, quem sabe?, nem um,\\
Nem haverá senão estrume de tantas conquistas futuras.\\
Não, não creio em mim.\\
Em todos os manicômios há doidos malucos com tantas certezas!\\
Eu, que não tenho nenhuma certeza, sou mais certo ou menos certo?\\
Não, nem em mim\ldots{}\\
Em quantas mansardas e não-mansardas do mundo\\
Não estão nesta hora gênios-para-si-mesmos sonhando?\\
Quantas aspirações altas e nobres e lúcidas -\\
Sim, verdadeiramente altas e nobres e lúcidas -,\\
E quem sabe se realizáveis,\\
Nunca verão a luz do sol real nem acharão ouvidos de gente?\\
O mundo é para quem nasce para o conquistar\\
E não para quem sonha que pode conquistá-lo, ainda que tenha razão.\\
Tenho sonhado mais que o que Napoleão fez.\\
Tenho apertado ao peito hipotético mais humanidades do que Cristo,\\
Tenho feito filosofias em segredo que nenhum Kant escreveu.\\
Mas sou, e talvez serei sempre, o da mansarda,\\
Ainda que não more nela;\\
Serei sempre o que não nasceu para isso;\\
Serei sempre só o que tinha qualidades;\\
Serei sempre o que esperou que lhe abrissem a porta ao pé de uma parede sem porta,\\
E cantou a cantiga do Infinito numa capoeira,\\
E ouviu a voz de Deus num poço tapado.\\
Crer em mim? Não, nem em nada.\\
Derrame-me a Natureza sobre a cabeça ardente\\
O seu sol, a sua chava, o vento que me acha o cabelo,\\
E o resto que venha se vier, ou tiver que vir, ou não venha.\\
Escravos cardíacos das estrelas,\\
Conquistamos todo o mundo antes de nos levantar da cama;\\
Mas acordamos e ele é opaco,\\
Levantamo-nos e ele é alheio,\\
Saímos de casa e ele é a terra inteira,\\
Mais o sistema solar e a Via Láctea e o Indefinido.

(Come chocolates, pequena;\\
Come chocolates!\\
Olha que não há mais metafísica no mundo senão chocolates.\\
Olha que as religiões todas não ensinam mais que a confeitaria.\\
Come, pequena suja, come!\\
Pudesse eu comer chocolates com a mesma verdade com que comes!\\
Mas eu penso e, ao tirar o papel de prata, que é de folha de estanho,\\
Deito tudo para o chão, como tenho deitado a vida.)

Mas ao menos fica da amargura do que nunca serei\\
A caligrafia rápida destes versos,\\
Pórtico partido para o Impossível.\\
Mas ao menos consagro a mim mesmo um desprezo sem lágrimas,\\
Nobre ao menos no gesto largo com que atiro\\
A roupa suja que sou, em rol, pra o decurso das coisas,\\
E fico em casa sem camisa.

(Tu que consolas, que não existes e por isso consolas,\\
Ou deusa grega, concebida como estátua que fosse viva,\\
Ou patrícia romana, impossivelmente nobre e nefasta,\\
Ou princesa de trovadores, gentilíssima e colorida,\\
Ou marquesa do século dezoito, decotada e longínqua,\\
Ou cocote célebre do tempo dos nossos pais,\\
Ou não sei quê moderno --- não concebo bem o quê ---\\
Tudo isso, seja o que for, que sejas, se pode inspirar que inspire!\\
Meu coração é um balde despejado.\\
Como os que invocam espíritos invocam espíritos invoco\\
A mim mesmo e não encontro nada.\\
Chego à janela e vejo a rua com uma nitidez absoluta.\\
Vejo as lojas, vejo os passeios, vejo os carros que passam,\\
Vejo os entes vivos vestidos que se cruzam,\\
Vejo os cães que também existem,\\
E tudo isto me pesa como uma condenação ao degredo,\\
E tudo isto é estrangeiro, como tudo.)

Vivi, estudei, amei e até cri,\\
E hoje não há mendigo que eu não inveje só por não ser eu.\\
Olho a cada um os andrajos e as chagas e a mentira,\\
E penso: talvez nunca vivesses nem estudasses nem amasses nem cresses\\
(Porque é possível fazer a realidade de tudo isso sem fazer nada disso);\\
Talvez tenhas existido apenas, como um lagarto a quem cortam o rabo\\
E que é rabo para aquém do lagarto remexidamente

Fiz de mim o que não soube\\
E o que podia fazer de mim não o fiz.\\
O dominó que vesti era errado.\\
Conheceram-me logo por quem não era e não desmenti, e perdi-me.\\
Quando quis tirar a máscara,\\
Estava pegada à cara.\\
Quando a tirei e me vi ao espelho,\\
Já tinha envelhecido.\\
Estava bêbado, já não sabia vestir o dominó que não tinha tirado.\\
Deitei fora a máscara e dormi no vestiário\\
Como um cão tolerado pela gerência\\
Por ser inofensivo\\
E vou escrever esta história para provar que sou sublime.

Essência musical dos meus versos inúteis,\\
Quem me dera encontrar-me como coisa que eu fizesse,\\
E não ficasse sempre defronte da Tabacaria de defronte,\\
Calcando aos pés a consciência de estar existindo,\\
Como um tapete em que um bêbado tropeça\\
Ou um capacho que os ciganos roubaram e não valia nada.

Mas o Dono da Tabacaria chegou à porta e ficou à porta.\\
Olho-o com o desconforto da cabeça mal voltada\\
E com o desconforto da alma mal-entendendo.\\
Ele morrerá e eu morrerei.\\
Ele deixará a tabuleta, eu deixarei os versos.\\
A certa altura morrerá a tabuleta também, os versos também.\\
Depois de certa altura morrerá a rua onde esteve a tabuleta,\\
E a língua em que foram escritos os versos.\\
Morrerá depois o planeta girante em que tudo isto se deu.\\
Em outros satélites de outros sistemas qualquer coisa como gente\\
Continuará fazendo coisas como versos e vivendo por baixo de coisas como tabuletas,

Sempre uma coisa defronte da outra,\\
Sempre uma coisa tão inútil como a outra,\\
Sempre o impossível tão estúpido como o real,\\
Sempre o mistério do fundo tão certo como o sono de mistério da superfície,\\
Sempre isto ou sempre outra coisa ou nem uma coisa nem outra.

Mas um homem entrou na Tabacaria (para comprar tabaco?)\\
E a realidade plausível cai de repente em cima de mim.\\
Semiergo-me enérgico, convencido, humano,\\
E vou tencionar escrever estes versos em que digo o contrário.

Acendo um cigarro ao pensar em escrevê-los\\
E saboreio no cigarro a libertação de todos os pensamentos.\\
Sigo o fumo como uma rota própria,\\
E gozo, num momento sensitivo e competente,\\
A libertação de todas as especulações\\
E a consciência de que a metafísica é uma consequência de estar mal disposto.

Depois deito-me para trás na cadeira\\
E continuo fumando.\\
Enquanto o Destino mo conceder, continuarei fumando.

(Se eu casasse com a filha da minha lavadeira\\
Talvez fosse feliz.)\\
Visto isto, levanto-me da cadeira. Vou à janela.\\
O homem saiu da Tabacaria (metendo troco na algibeira das calças?).\\
Ah, conheço-o; é o Esteves sem metafísica.\\
(O Dono da Tabacaria chegou à porta.)\\
Como por um instinto divino o Esteves voltou-se e viu-me.\\
Acenou-me adeus, gritei-lhe Adeus ó Esteves!, e o universo\\
Reconstruiu-se-me sem ideal nem esperança, e o Dono da Tabacaria sorriu.
\end{verse}

%===============================================================================
\part{Antonio Machado}

%-------------------------------------------------------------------------------
\chapter{Retrato}

\begin{verse}
Mi infancia son recuerdos de un patio de Sevilla,\\
y un huerto claro donde madura el limonero;\\
mi juventud, veinte años en tierras de Castilla;\\
mi historia, algunos casos que recordar no quiero.

Ni un seductor Mañara, ni un Bradomín he sido\\
?`ya conocéis mi torpe aliño indumentario?,\\
más recibí la flecha que me asignó Cupido,\\
y amé cuanto ellas puedan tener de hospitalario.

Hay en mis venas gotas de sangre jacobina,\\
pero mi verso brota de manantial sereno;\\
y, más que un hombre al uso que sabe su doctrina,\\
soy, en el buen sentido de la palabra, bueno.

Adoro la hermosura, y en la moderna estética\\
corté las viejas rosas del huerto de Ronsard;\\
mas no amo los afeites de la actual cosmética,\\
ni soy un ave de esas del nuevo gay-trinar.

Desdeño las romanzas de los tenores huecos\\
y el coro de los grillos que cantan a la luna.\\
A distinguir me paro las voces de los ecos,\\
y escucho solamente, entre las voces, una.

?`Soy clásico o romántico? No sé. Dejar quisiera\\
mi verso, como deja el capitán su espada:\\
famosa por la mano viril que la blandiera,\\
no por el docto oficio del forjador preciada.

Converso con el hombre que siempre va conmigo\\
?`quien habla solo espera hablar a Dios un día?;\\
mi soliloquio es plática con ese buen amigo\\
que me enseñó el secreto de la filantropía.

Y al cabo, nada os debo; debéisme cuanto he escrito.\\
A mi trabajo acudo, con mi dinero pago\\
el traje que me cubre y la mansión que habito,\\
el pan que me alimenta y el lecho en donde yago.

Y cuando llegue el día del último vïaje,\\
y esté al partir la nave que nunca ha de tornar,\\
me encontraréis a bordo ligero de equipaje,\\
casi desnudo, como los hijos de la mar.
\end{verse}

%-------------------------------------------------------------------------------
\chapter{Nuevas Canciones -- IV}

\begin{verse}
Esta luz de Sevilla\ldots{} Es el palacio\\
donde nací, con su rumor de fuente.\\
Mi padre, en su despacho. ---La alta frente,\\
la breve mosca, y el bigote lacio---.

Mi padre, aún joven. Lee, escribe, hojea\\
sus libros y medita. Se levanta;\\
va hacia la puerta del jardín. Pasea.\\
A veces habla solo, a veces canta.

Sus grandes ojos de mirar inquieto\\
ahora vagar parecen, sin objeto\\
donde puedan posar, en el vacío.

Ya escapan de su ayer a su mañana;\\
ya miran en el tiempo, ¡padre mío!,\\
piadosamente mi cabeza cana.
\end{verse}

%===============================================================================
\part{Charles Baudelaire}

%-------------------------------------------------------------------------------
\chapter{LXXIV -- La Cloche fêlée}

\begin{verse}
Il est amer et doux, pendant les nuits d'hiver,\\
D'écouter, près du feu qui palpite et qui fume,\\
Les souvenirs lointains lentement s'élever\\
Au bruit des carillons qui chantent dans la brume,

Bienheureuse la cloche au gosier vigoureux\\
Qui, malgré sa vieillesse, alerte et bien portante,\\
Jette fidèlement son cri religieux,\\
Ainsi qu'un vieux soldat qui veille sous la tente!

Moi, mon âme est fêlée, et lorsqu'en ses ennuis\\
Elle veut de ses chants peupler l'air froid des nuits,\\
Il arrive souvent que sa voix affaiblie

Semble le râle épais d'un blessé qu'on oublie\\
Au bord d'un lac de sang, sous un grand tas de morts,\\
Et qui meurt, sans bouger, dans d'immenses efforts.
\end{verse}

%===============================================================================
\part{William Cowper}

%-------------------------------------------------------------------------------
\chapter{On an ugly fellow}

\begin{verse}
Beware, my friend! of crystal brook,\\
Or fountain, lest that hideous hook,\\
Thy nose, thou chance to see;\\
Narcissus' fate would then be thine,\\
And self-detested thou wouldst pine,\\
As self-enamour'd he.
\end{verse}

~

\begin{verse} \footnotesize
Evita, amigo, evita debruçar-te\\
Sobre o cristal de um cristalino veio,\\
Senão, como Narciso, irás matar-te,\\
Não por te veres belo, mas tão feio.
\end{verse}

%===============================================================================
\part{W. B. Yeats}

%-------------------------------------------------------------------------------
\chapter{The Four Ages of Man}

\begin{verse}
He with body waged a fight,\\
But body won; it walks upright.\\
Then he struggled with the heart;\\
Innocence and peace depart.\\
Then he struggled with the mind;\\
His proud heart he left behind.\\
Now his wars on God begin;\\
At stroke of midnight God shall win.
\end{verse}

%-------------------------------------------------------------------------------
\chapter{The Second Coming}

\begin{verse}
Turning and turning in the widening gyre\\
The falcon cannot hear the falconer;\\
Things fall apart; the centre cannot hold;\\
Mere anarchy is loosed upon the world,\\
The blood-dimmed tide is loosed, and everywhere\\
The ceremony of innocence is drowned;\\
The best lack all conviction, while the worst\\
Are full of passionate intensity.

Surely some revelation is at hand;\\
Surely the Second Coming is at hand.\\
The Second Coming! Hardly are those words out\\
When a vast image out of Spiritus Mundi\\
Troubles my sight: a waste of desert sand;\\
A shape with lion body and the head of a man,\\
A gaze blank and pitiless as the sun,\\
Is moving its slow thighs, while all about it\\
Wind shadows of the indignant desert birds.

~~

The darkness drops again but now I know\\
That twenty centuries of stony sleep\\
Were vexed to nightmare by a rocking cradle,\\
And what rough beast, its hour come round at last,\\
Slouches towards Bethlehem to be born?
\end{verse}

%-------------------------------------------------------------------------------
\chapter{When You Are Old}

\begin{verse}
When you are old and grey and full of sleep,\\
And nodding by the fire, take down this book,\\
And slowly read, and dream of the soft look\\
Your eyes had once, and of their shadows deep;

How many loved your moments of glad grace,\\
And loved your beauty with love false or true,\\
But one man loved the pilgrim soul in you,\\
And loved the sorrows of your changing face;

And bending down beside the glowing bars,\\
Murmur, a little sadly, how Love fled\\
And paced upon the mountains overhead\\
And hid his face amid a crowd of stars.
\end{verse}

%===============================================================================
\part{Charles Best}

%-------------------------------------------------------------------------------
\chapter{A Sonnet of the Moon}

\begin{verse}
Look how the pale queen of the silent night\\
Doth cause the ocean to attend upon her,\\
And he, as long as she is in his sight,\\
With her full tide is ready her to honor.

But when the silver waggon of the moon\\
Is mounted up so high he cannot follow,\\
The sea calls home his crystal waves to moan,\\
And with low ebb doth manifest his sorrow.

So you that are the sovereign of my heart\\
Have all my joys attending on your will;\\
My joys low-ebbing when you do depart,

When you return their tide my heart doth fill.\\
\hspace{3em}So as you come and as you do depart,\\
\hspace{3em}Joys ebb and flow within my tender heart.
\end{verse}

%===============================================================================
\part{Shakespeare}

%-------------------------------------------------------------------------------
\chapter{Hamlet}

\begin{verse}
To be, or not to be--that is the question:\\
Whether 'tis nobler in the mind to suffer\\
The slings and arrows of outrageous fortune\\
Or to take arms against a sea of troubles\\
And by opposing end them. To die, to sleep--\\
No more--and by a sleep to say we end\\
The heartache, and the thousand natural shocks\\
That flesh is heir to. 'Tis a consummation\\
Devoutly to be wished. To die, to sleep--\\
To sleep--perchance to dream: ay, there's the rub,\\
For in that sleep of death what dreams may come\\
When we have shuffled off this mortal coil,\\
Must give us pause. There's the respect\\
That makes calamity of so long life.\\
For who would bear the whips and scorns of time,\\
Th' oppressor's wrong, the proud man's contumely\\
The pangs of despised love, the law's delay,\\
The insolence of office, and the spurns\\
That patient merit of th' unworthy takes,\\
When he himself might his quietus make\\
With a bare bodkin? Who would fardels bear,\\
To grunt and sweat under a weary life,\\
But that the dread of something after death,\\
The undiscovered country, from whose bourn\\
No traveller returns, puzzles the will,\\
And makes us rather bear those ills we have\\
Than fly to others that we know not of?\\
Thus conscience does make cowards of us all,\\
And thus the native hue of resolution\\
Is sicklied o'er with the pale cast of thought,\\
And enterprise of great pitch and moment\\
With this regard their currents turn awry\\
And lose the name of action. -- Soft you now,\\
The fair Ophelia! -- Nymph, in thy orisons\\
Be all my sins remembered.
\end{verse}

%===============================================================================
\part{Lord Tennyson}

%-------------------------------------------------------------------------------
\chapter{Ring out, Wild Bells}

\begin{verse}
Ring out, wild bells, to the wild sky,\\
The flying cloud, the frosty light;\\
The year is dying in the night;\\
Ring out, wild bells, and let him die.

Ring out the old, ring in the new,\\
Ring, happy bells, across the snow:\\
The year is going, let him go;\\
Ring out the false, ring in the true.

Ring out the grief that saps the mind,\\
For those that here we see no more,\\
Ring out the feud of rich and poor,\\
Ring in redress to all mankind.

Ring out a slowly dying cause,\\
And ancient forms of party strife;\\
Ring in the nobler modes of life,\\
With sweeter manners, purer laws.

Ring out the want, the care, the sin,\\
The faithless coldness of the times;\\
Ring out, ring out thy mournful rhymes,\\
But ring the fuller minstrel in.

Ring out false pride in place and blood,\\
The civic slander and the spite;\\
Ring in the love of truth and right,\\
Ring in the common love of good.

Ring out old shapes of foul disease,\\
Ring out the narrowing lust of gold;\\
Ring out the thousand wars of old,\\
Ring in the thousand years of peace.

Ring in the valiant man and free,\\
The larger heart the kindlier hand;\\
Ring out the darkness of the land,\\
Ring in the Christ that is to be.
\end{verse}

%===============================================================================
\part{Rudyard Kipling}

%-------------------------------------------------------------------------------
\chapter{If---}

\begin{verse}
If you can keep your head when all about you\\
\hspace{1em}Are losing theirs and blaming it on you,\\
If you can trust yourself when all men doubt you,\\
\hspace{1em}But make allowance for their doubting too;\\
If you can wait and not be tired by waiting,\\
\hspace{1em}Or being lied about, don't deal in lies,\\
Or being hated, don't give way to hating,\\
\hspace{1em}And yet don't look too good, nor talk too wise:

If you can dream---and not make dreams your master;\\
\hspace{1em}If you can think---and not make thoughts your aim;\\
If you can meet with Triumph and Disaster\\
\hspace{1em}And treat those two impostors just the same;\\
If you can bear to hear the truth you've spoken\\
\hspace{1em}Twisted by knaves to make a trap for fools,\\
Or watch the things you gave your life to, broken,\\
\hspace{1em}And stoop and build 'em up with worn-out tools:

If you can make one heap of all your winnings\\
\hspace{1em}And risk it on one turn of pitch-and-toss,\\
And lose, and start again at your beginnings\\
\hspace{1em}And never breathe a word about your loss;\\
If you can force your heart and nerve and sinew\\
\hspace{1em}To serve your turn long after they are gone,\\
And so hold on when there is nothing in you\\
\hspace{1em}Except the Will which says to them: `Hold on!'

If you can talk with crowds and keep your virtue,\\
\hspace{1em}Or walk with Kings---nor lose the common touch,\\
If neither foes nor loving friends can hurt you,\\
\hspace{1em}If all men count with you, but none too much;\\
If you can fill the unforgiving minute\\
\hspace{1em}With sixty seconds' worth of distance run,\\
Yours is the Earth and everything that's in it,\\
\hspace{1em}And---which is more---you'll be a Man, my son!
\end{verse}

\end{document}
